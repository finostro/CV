\documentclass{article}
\usepackage{bubblecv}
\usepackage{bibentry}
\usepackage[square, numbers, sort]{natbib}



\begin{document}


\begin{cv}[avatar]{Felipe Inostroza Ferrari}{Electrical Engineer, PhD.}


\cvsection[summary]{Profile}  %-----------------------------------------------------------

Felipe obtained the degree of Electrical Engineer and Master of Electrical Engineering on 2014  and got the degree of Doctor of Electrical Engineering on 2023. His reaserch interests are Robotics, Autonomous Vehicles, SLAM and sensor data processing in general. 
He is capable of choosing sensors apply algorithms to process the data they produce and design and implement algorithms that use the process data to make decisions.  

\cvsection[work]{Work Experience}  %------------------------------------------------------

\begin{cvevent}[2019][present]
    \cvname{Researcher}
    \cvdescription{Advanced Mining Technology Center (AMTC), Universidad de Chile}
    I currently work at AMTC in the University of Chile designing and developing mapping and localizations solutions for vehicles operating in underground mining as well as in autonomous navigation in general.  
\end{cvevent}

\begin{cvevent}[2014][2021]
    \cvname{Auxiliary Teacher}
    \cvdescription{Department of Electrical Engineering, Universidad de Chile}
    I was tasked with making auxiliary classes, solving practical problems and guiding tutorials in the following courses
    \begin{itemize}
        \item Cumputational Inteligence and Robotics Laboratory (2014-2021),
        \item Robotics, Sensing and Autonomous Systems(2013-2019).
        \item System Control Fundamentals (2013-2014).
        \item Fault Diagnosis and Prognosis (2013-2013).
    \end{itemize}
\end{cvevent}


\cvsection[education]{Education}  %------------------------------------------------------

\begin{cvevent}[2014][2023]
    \cvname{Doctorate in Electrical Engineering}
    \cvdescription{Department of Electrical Engineering, Universidad de Chile}
My thesis topic was to apply modern Random Finite Sets (RFSs) theory to the problem of Simulatanous Localization and Mapping (SLAM). Modifications to existing state of the art algorithms were proposed and tested. Additionally, a new RFS-based algorithm was developed and tested with stereo cameras.
\end{cvevent}

\begin{cvevent}[2012][2014]
    \cvname{Master of Electrical Engineering}
    \cvdescription{Department of Electrical Engineering, Universidad de Chile}
Two existing SLAM algorithm were modified and tested with 2D LIDAR data. The proposed modification was shown improve performance for both algorithms.
\end{cvevent}
\begin{cvevent}[2007][2014]
    \cvname{Electrical Engineering}
    \cvdescription{Department of Electrical Engineering, Universidad de Chile}
\end{cvevent}
\begin{cvevent}[2007][2014]
    \cvname{Primary and Secondary Education}
    \cvdescription{Colegio Alemán de Temuco}
\end{cvevent}


% \bibliographystyle{}
%\bibliographystyle{IEEEtranN}
\bibliographystyle{unsrtnat}



\cvsidebar %-----------------------------------------------------------------------------


\cvsection[contact]{Contact}  %----------------------------------------------------------

\begin{cvitem}[Envelope][4]
    \textbf{Email}\\
    \href{mailto:pepeinostroza@gmail.com}{\texttt{pepeinostroza@gmail.com}}
\end{cvitem}

\cvseparator[3]
\begin{cvitem}[Phone][4]
    \textbf{Phone}\\
    \href{tel:+56976499574}{\texttt{+56976499574}}
\end{cvitem}

\cvseparator[3]
\begin{cvitem}[Home][4]
    \textbf{Home Address}\\
    Tucapel 233, depto. 2107, Ñuñoa\\ Santiago, Chile
\end{cvitem}

\cvseparator[3]
\begin{cvitem}[Globe][4]
    \textbf{Links}\\
    \href{https://www.researchgate.net/profile/Felipe-Inostroza-4}{\texttt{ResearchGate}} \\
    \href{https://www.linkedin.com/in/felipe-inostroza-ferrari-a0852035/}{\texttt{LinkedIn}}
\end{cvitem}


\cvsection[skills]{Skills}  %-----------------------------------------------------------
\cvskill{C++}{Advanced}{0.7}
\cvskill{ROS}{Advanced}{0.7}
\cvskill{Python}{Intermediate}{0.4}
\cvskill{Git}{Intermediate}{0.4}
\cvseparator
\begin{cvitem}
    Windows, Linux, Mac OS X 
\end{cvitem}

\cvseparator

\begin{cvitem}
    C, C++, Python, MATLAB, Git
\end{cvitem}

\cvseparator
\begin{cvitem}
    Microsoft Office, Excel, OpenOffice, \LaTeX 
\end{cvitem}




\cvsection[languages]{Languages}  %--------------------------------------------------------

\cvskill{Spanish}{Native}{1.0}
\cvskill{English}{Fluent}{0.8}
\cvskill{German}{Basic}{0.1}









\end{cv}

%\bibliography{references}
\nobibliography{references}
\cvsection[publications]{Publications}
\begin{cvevent}
    \cvname{Journal Articles}
    \begin{itemize}
    \item \bibentry{inostroza2018modeling}
    \item \bibentry{leung2017chilean}
    \item \bibentry{leung2017relatingTSP}
    \item \bibentry{cola_metric_tro}
    \item \bibentry{glmbslam_sensors_Moratuwage}
    \item \bibentry{leung_2016_taes_multifeature}
    \end{itemize}
\end{cvevent}
\begin{cvevent}
    \cvname{Conference Papers}
    \begin{itemize}
    \item \bibentry{Leung2013}
    \item \bibentry{inostroza2014semantic}
    \item \bibentry{leung2014evaluating}
    \item \bibentry{inostroza2015incorporating}
    \item \bibentry{leung2015generalizing}
    \item \bibentry{glmb_slam_moratuwage}
    \item \bibentry{inostroza_fusion_2018_addressing}
    \item \bibentry{cola_metric_fusion_2015}
    \end{itemize}
\end{cvevent}

% \cvfootnote{
%     \tiny I agree to the processing of personal data provided in this document for realizing the recruitment process pursuant to the Personal Data Protection Act of 10 May 2018 (Journal of Laws 2018, item 1000) and in agreement with Regulation (EU) 2016/679 of the European Parliament and of the Council of 27 April 2016 on the protection of natural persons with regard to the processing of personal data and on the free movement of such data, and repealing Directive 95/46/EC (General Data Protection Regulation)
% }

\end{document}
