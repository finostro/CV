\documentclass{article}
\usepackage{bubblecv}
\usepackage{bibentry}
\usepackage[square, numbers, sort]{natbib}


\nobibliography*

\begin{document}


\begin{cv}[avatar]{Felipe Inostroza Ferrari}{Ingeniero Eléctrico, PhD.}


\cvsection[summary]{Perfil}  %-----------------------------------------------------------

 Felipe obtuvo el grado de Ingeniero Eléctrico y Magíster en Ingeniería Eléctrica el 2014 y obtuvo el grado de Doctor en Ingeniería Eléctrica el 2023. Sus temas de interés son la Robótica, Vehículos Autónomos y el procesamiento de datos de sensores en general. Es capaz de elegir y aplicar algoritmos de procesamiento de datos y/o usar los resultados para generar acciones autónomas. 

\cvsection[work]{Experiencia}  %------------------------------------------------------

\begin{cvevent}[2019][present]
    \cvname{Investigador post-doctoral}
    \cvdescription{Advanced Mining Technology Center (AMTC), Universidad de Chile}
    Actualmente trabajo como investigador en el AMTC, de la Universidad de Chile, diseñando y desarrollando sistemas de mapeo y localización en túneles para vehículos mineros, y en navegación de equipos autónomos en general. 
\end{cvevent}

\begin{cvevent}[2014][2021]
    \cvname{Profesor Auxiliar }
    \cvdescription{Departamento de Ingeniería Eléctrica, Universidad de Chile}
    Me correspondía realizar clases auxiliares en que se resuelven problemas y guiar tutoriales en los siguientes cursos:
    \begin{itemize}
        \item Laboratorio de Inteligencia Computacional y Robótica (2014-2021),
        \item Robotics, Sensing and Autonomous Systems(2013-2019).
        \item Fundamentos de Control de Sistemas (2013-2014).
        \item Diagnóstico Y Pronóstico de Fallas (2013-2013).
    \end{itemize}
\end{cvevent}


\cvsection[education]{Educación}  %------------------------------------------------------

\begin{cvevent}[2014][2023]
    \cvname{Doctorado en Ingeniería Eléctrica}
    \cvdescription{Departamento de Ingeniería Eléctrica, Universidad de Chile}
Mi tema de tesis fue aplicar técnicas modernas basadas en Conjuntos Aleatorios Finitos a el problema de localización y construcción de mapas simultanea. Se introdujeron modificaciones a algoritmos del estado del arte y se introdujo un algoritmo nuevo para resolver el problema en el caso de cámaras estéreo.
\end{cvevent}

\begin{cvevent}[2012][2014]
    \cvname{Magíster en Ingeniería Civil Eléctrica}
    \cvdescription{Departamento de Ingeniería Eléctrica, Universidad de Chile}
El tema de tesis fue la implementación de un algoritmo de localización y construcción de  mapas simultanea basado en conjunto aleatorios finitos, con un sensor de rango láser 2D. 
\end{cvevent}
\begin{cvevent}[2007][2014]
    \cvname{Ingeniería Civil Eléctrica}
    \cvdescription{Departamento de Ingeniería Eléctrica, Universidad de Chile}
\end{cvevent}
\begin{cvevent}[2007][2014]
    \cvname{Educación Básica y Media}
    \cvdescription{Colegio Alemán de Temuco}
\end{cvevent}


% \bibliographystyle{}
%\bibliographystyle{IEEEtranN}
\bibliographystyle{unsrtnat}



\cvsidebar %-----------------------------------------------------------------------------


\cvsection[contact]{Contacto}  %----------------------------------------------------------

\begin{cvitem}[Envelope][4]
    \textbf{Email}\\
    \href{mailto:pepeinostroza@gmail.com}{\texttt{pepeinostroza@gmail.com}}
\end{cvitem}

\cvseparator[3]
\begin{cvitem}[Phone][4]
    \textbf{Phone}\\
    \href{tel:+56976499574}{\texttt{+56976499574}}
\end{cvitem}

\cvseparator[3]
\begin{cvitem}[Home][4]
    \textbf{Direccion}\\
    Tucapel 233, depto. 2107, Ñuñoa\\ Santiago, Chile
\end{cvitem}

\cvseparator[3]
\begin{cvitem}[Globe][4]
    \textbf{Links}\\
    \href{https://www.researchgate.net/profile/Felipe-Inostroza-4}{\texttt{ResearchGate}} \\
    \href{https://www.linkedin.com/in/felipe-inostroza-ferrari-a0852035/}{\texttt{LinkedIn}}
\end{cvitem}


\cvsection[skills]{Skills}  %-----------------------------------------------------------
\cvskill{C++}{Advanced}{0.7}
\cvskill{ROS}{Advanced}{0.7}
\cvskill{Python}{Intermediate}{0.4}
\cvskill{Git}{Intermediate}{0.4}
\cvseparator
\begin{cvitem}
    Windows, Linux, Mac OS X 
\end{cvitem}

\cvseparator

\begin{cvitem}
    C, C++, Python, MATLAB, Git
\end{cvitem}

\cvseparator
\begin{cvitem}
    Microsoft Office, Excel, OpenOffice, \LaTeX 
\end{cvitem}




\cvsection[languages]{Idiomas}  %--------------------------------------------------------

\cvskill{Español}{Nativo}{1.0}
\cvskill{Inglés}{Alto}{0.8}
\cvskill{Alemán}{Basico}{0.1}









\end{cv}

%\bibliography{references}
\cvsection[publications]{Publicaciones}
\begin{cvevent}
    \cvname{Publicaciones en Revistas}
    \begin{itemize}
    \item \bibentry{inostroza2018modeling}
    \item \bibentry{leung2017chilean}
    \item \bibentry{leung2017relatingTSP}
    \item \bibentry{cola_metric_tro}
    \item \bibentry{glmbslam_sensors_Moratuwage}
    \item \bibentry{leung_2016_taes_multifeature}
    \end{itemize}
\end{cvevent}
\begin{cvevent}
    \cvname{Publicaciones en Conferencias}
    \begin{itemize}
    \item \bibentry{Leung2013}
    \item \bibentry{inostroza2014semantic}
    \item \bibentry{leung2014evaluating}
    \item \bibentry{inostroza2015incorporating}
    \item \bibentry{leung2015generalizing}
    \item \bibentry{glmb_slam_moratuwage}
    \item \bibentry{inostroza_fusion_2018_addressing}
    \item \bibentry{cola_metric_fusion_2015}
    \end{itemize}
\end{cvevent}

% \cvfootnote{
%     \tiny I agree to the processing of personal data provided in this document for realizing the recruitment process pursuant to the Personal Data Protection Act of 10 May 2018 (Journal of Laws 2018, item 1000) and in agreement with Regulation (EU) 2016/679 of the European Parliament and of the Council of 27 April 2016 on the protection of natural persons with regard to the processing of personal data and on the free movement of such data, and repealing Directive 95/46/EC (General Data Protection Regulation)
% }

\end{document}
