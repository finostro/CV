\documentclass{article}
\usepackage{bubblecv}
\usepackage{bibentry}
\usepackage[square, numbers, sort]{natbib}



\begin{document}


\begin{cv}[avatar]{Felipe Inostroza Ferrari}{Ingeniero Eléctrico, PhD.}


	\cvsection[summary]{Perfil}  %-----------------------------------------------------------

	Felipe obtuvo el grado de Ingeniero Eléctrico y Magíster en Ingeniería Eléctrica el 2014 y obtuvo el grado de Doctor en Ingeniería Eléctrica el 2023. Sus temas de interés son la Robótica, Vehículos Autónomos y el procesamiento de datos de sensores en general. Es capaz de elegir y aplicar algoritmos de procesamiento de datos y/o usar los resultados para generar acciones autónomas.

	\cvsection[work]{Experiencia}  %------------------------------------------------------

	\begin{cvevent}[2020][presente]
		\cvname{Investigador Asociado}
		\cvdescription{Advanced Mining Technology Center (AMTC), Universidad de Chile}
		Actualmente trabajo como investigador en el AMTC, de la Universidad de Chile, diseñando y desarrollando sistemas de mapeo y localización en túneles para vehículos mineros, y en navegación de equipos autónomos en general. Desde 2024 que me encuentro a cargo del Área técnica del proyecto de Automatización de carguío y navegación de vehículos Load-Haul-Dump (LHD).
	\end{cvevent}

	\begin{cvevent}[2014][2021]
		\cvname{Profesor Auxiliar }
		\cvdescription{Departamento de Ingeniería Eléctrica, Universidad de Chile}
		Me correspondía realizar clases auxiliares en que se resuelven problemas y guiar tutoriales en los siguientes cursos:
		\begin{itemize}
			\item EL5206 Laboratorio de Inteligencia Computacional y Robótica (2014-2021),
			\item EL7031 Robotics, Sensing and Autonomous Systems (2013-2019),
			\item EL4004 Fundamentos de Control de Sistemas (2013-2014).
		\end{itemize}
	\end{cvevent}

	\begin{cvevent}[2013,2015]
		\cvname{Ayudante }
		\cvdescription{Departamento de Ingeniería Eléctrica, Universidad de Chile}
		Me correspondía ayudar en la evaluación de los estudiantes, principalmente corrigiendo pruebas e informes, pero también evaluando presentaciones, en los siguientes cursos:
		\begin{itemize}
			\item EL7031 Robotics, Sensing and Autonomous Systems(2015),
			\item EL7014 Diagnóstico Y Pronóstico de Fallas (2013-2013).
		\end{itemize}
	\end{cvevent}

	\begin{cvevent}[2015][2017]
		\cvname{Voluntario - Talleres de Robótica}
		\cvdescription{Fundación Mustakis}
		Participé como voluntario en talleres de robótica educacional para estudiantes desde séptimo básico hasta tercero medio.
		En estos talleres los estudiantes aprenden conceptos de programación, además de fortalecer competencias como trabajo colaborativo, perseverancia, comunicación y liderazgo. Mi rol en los talleres fue el de dar consejos y soporte a los estudiantes mientras ellos aprendían programando robots con Arduino para realizar distintas tareas.
	\end{cvevent}


	% \bibliographystyle{}
	%\bibliographystyle{IEEEtranN}
	\bibliographystyle{unsrtnat}



	\cvsidebar %-----------------------------------------------------------------------------


	\cvsection[contact]{Contacto}  %----------------------------------------------------------

	\begin{cvitem}[Envelope][4]
		\textbf{Email}\\
		\href{mailto:pepeinostroza@gmail.com}{\texttt{pepeinostroza@gmail.com}}
	\end{cvitem}

	\cvseparator[3]
	\begin{cvitem}[Phone][4]
		\textbf{Phone}\\
		\href{tel:+56976499574}{\texttt{+56976499574}}
	\end{cvitem}

	\cvseparator[3]
	\begin{cvitem}[Home][4]
		\textbf{Direccion}\\
		Tucapel 233, depto. 2107, Ñuñoa\\ Santiago, Chile
	\end{cvitem}

	\cvseparator[3]
	\begin{cvitem}[Globe][4]
		\textbf{Links}\\
		\href{https://www.researchgate.net/profile/Felipe-Inostroza-4}{\texttt{ResearchGate}} \\
		\href{https://www.linkedin.com/in/felipe-inostroza-ferrari-a0852035/}{\texttt{LinkedIn}}
	\end{cvitem}


	\cvsection[skills]{Habilidades}  %-----------------------------------------------------------
	\cvskill{C++}{Advanced}{0.8}
	\cvskill{ROS}{Advanced}{0.8}
	\cvskill{Python}{Advanced}{0.7}
	\cvskill{Git}{Intermediate}{0.4}
	\cvseparator
	\begin{cvitem}
		Windows, Linux, Mac OS X
	\end{cvitem}

	\cvseparator

	\begin{cvitem}
		C, C++, Python, MATLAB, Git
	\end{cvitem}

	\cvseparator
	\begin{cvitem}
		Microsoft Office, Excel, OpenOffice, \LaTeX
	\end{cvitem}




	\cvsection[languages]{Idiomas}  %--------------------------------------------------------

	\cvskill{Español}{Nativo}{1.0}
	\cvskill{Inglés}{Alto}{0.8}
	\cvskill{Alemán}{Basico}{0.1}









\end{cv}

\clearpage
\cvsection[education]{Educación}  %------------------------------------------------------

\begin{cvevent}[2014][2023]
	\cvname{Doctorado en Ingeniería Eléctrica}
	\cvdescription{Departamento de Ingeniería Eléctrica, Universidad de Chile}
	Mi tema de tesis fue aplicar técnicas modernas basadas en Conjuntos Aleatorios Finitos a el problema de localización y construcción de mapas simultanea. Se introdujeron modificaciones a algoritmos del estado del arte y se introdujo un algoritmo nuevo para resolver el problema en el caso de cámaras estéreo.
\end{cvevent}

\begin{cvevent}[2012][2014]
	\cvname{Magíster en Ingeniería Civil Eléctrica}
	\cvdescription{Departamento de Ingeniería Eléctrica, Universidad de Chile}
	El tema de tesis fue la implementación de un algoritmo de localización y construcción de  mapas simultanea basado en conjunto aleatorios finitos, con un sensor de rango láser 2D.
\end{cvevent}
\begin{cvevent}[2007][2014]
	\cvname{Ingeniería Civil Eléctrica}
	\cvdescription{Departamento de Ingeniería Eléctrica, Universidad de Chile}
\end{cvevent}
\begin{cvevent}[2007][2014]
	\cvname{Educación Básica y Media}
	\cvdescription{Colegio Alemán de Temuco}
\end{cvevent}

\cvsection[awards]{Premios}
\begin{cvevent}[2024]
	\cvname{ Mejor tesis de Doctorado}
	\cvdescription{Departamento de Ingeniería Eléctrica, Universidad de Chile}
	Premio otorgado por el Departamento de Ingeniería Eléctrica (DIE) a la mejor tesis de doctorado, dentro de las tesis aceptadas ese año.
\end{cvevent}
\begin{cvevent}[2013]
	\cvname{Best Paper Award}
	\cvdescription{International Conference on Control, Automation
		and Information Sciences (ICCAIS) 2013}
	Premio al mejor paper en la conferencia ``International Conference on Control, Automation
	and Information Sciences". El tema del paper consistió en una modificación a un algoritmo de
	SLAM basado en RFS.
\end{cvevent}
\begin{cvevent}[2011]
	\cvname{Alumno Destacado}
	\cvdescription{Facultad de Ciencias Físicas y Matemáticas, Universidad de Chile}
	Esta distinción es otorgada a los alumnos con un promedio ponderado superior a 5.7, o dentro
	del mejor 6\%, considerando a todos los alumnos que entraron a la Carrera en el mismo año.
\end{cvevent}

\begin{cvevent}[2007][2009]
	\cvname{Alumno Destacado}
	\cvdescription{Facultad de Ciencias Físicas y Matemáticas, Universidad de Chile}
	Esta distinción es otorgada a los alumnos con un promedio ponderado superior a 5.7, o dentro
	del mejor 6\%, considerando a todos los alumnos que entraron a la Carrera en el mismo año.
\end{cvevent}




\clearpage
%\bibliography{references}
\nobibliography{references}
\cvsection[publications]{Publicaciones}
\begin{cvevent}
	\cvname{Publicaciones en Revistas}
	\begin{itemize}
		\item \bibentry{inostroza_2024_jvsslam}
		\item \bibentry{inostroza_2023_robust_localization}
		\item \bibentry{cardenas_2023_autonomous_loading}
		\item \bibentry{inostroza2018modeling}
		\item \bibentry{leung2017chilean}
		\item \bibentry{leung2017relatingTSP}
		\item \bibentry{cola_metric_tro}
		\item \bibentry{glmbslam_sensors_Moratuwage}
		\item \bibentry{leung_2016_taes_multifeature}
	\end{itemize}
\end{cvevent}
\begin{cvevent}
	\cvname{Publicaciones en Conferencias}
	\begin{itemize}
		\item \bibentry{Leung2013}
		\item \bibentry{inostroza2014semantic}
		\item \bibentry{leung2014evaluating}
		\item \bibentry{inostroza2015incorporating}
		\item \bibentry{leung2015generalizing}
		\item \bibentry{glmb_slam_moratuwage}
		\item \bibentry{inostroza_fusion_2018_addressing}
		\item \bibentry{cola_metric_fusion_2015}
	\end{itemize}
\end{cvevent}

\begin{cvevent}
	\cvname{Registro de Software}
	\begin{itemize}
		\item RFS SLAM-TRACK 1.0: Software para seguimiento de objetivos múltiples, el cual provee filtros que pueden proporcionar estimaciones óptimas del número y estado de múltiples objetivos, basados en mediciones de sensores ruidosos, que contienen tanto mediciones como errores de detección.
	\end{itemize}
\end{cvevent}

% \cvfootnote{
%     \tiny I agree to the processing of personal data provided in this document for realizing the recruitment process pursuant to the Personal Data Protection Act of 10 May 2018 (Journal of Laws 2018, item 1000) and in agreement with Regulation (EU) 2016/679 of the European Parliament and of the Council of 27 April 2016 on the protection of natural persons with regard to the processing of personal data and on the free movement of such data, and repealing Directive 95/46/EC (General Data Protection Regulation)
% }

\end{document}
